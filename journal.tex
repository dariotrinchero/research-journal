%=========================================================================================
% This is journal.tex, an example of using the journal.cls class, written by
% Dario Trinchero (dario.trinchero@pm.me).
%=========================================================================================

\documentclass[% see journal.cls for more documentation on class options
	raggedright, % ragged-right alignment (instead of justified)
	%widetoc, % wide table-of-contents (instead of 2-column)
]{journal}

%-----------------------------------------------------------------------------------------
% Custom definitions / imports
%-----------------------------------------------------------------------------------------

% TODO add your custom definitions & imports here

%-----------------------------------------------------------------------------------------
% Journal setup
%-----------------------------------------------------------------------------------------

\name{Research Journal}
\project{PhD Journalism}
\subject{The use of LaTeX typesetting for journalism} % only in PDF file metadata
\author{Alice Atwood \and Bob Benston}

%-----------------------------------------------------------------------------------------
% Front matter
%-----------------------------------------------------------------------------------------

% The following line mocks current date so that template compiles to a
% consistent output for demonstration purposes.
\SetDate[10/01/2023] % TODO DELETE THIS LINE

\begin{document}
\maketitle

% optional table of contents
\tableofcontents % 2-column layout unless "widetoc" class option given

% optional calendar with links to entries & current day highlighted
\calendar[11-01]{2022}{2025} % syntax is \calendar[first-day]{start-year}{end-year}

% optional "Planned Timeline" section with Gantt chart(s) from figures/gantt.tex
\timelines % \timelines[path] gives custom path to Gantt chart definitions

% optional list of unresolved questions & TODOs
\listofissues % see entries below for syntax of adding & resolving issues

%-----------------------------------------------------------------------------------------
% Entries
%-----------------------------------------------------------------------------------------

\logstart{2022-12-23} % date for first entry; use yyyy-mm-dd format

\logday % add entry
Add entries with \verb|\logday|. You can skip days using \verb|\logday[<days>]|.
Days are automatically \hyperref[2022-12-02]{labeled} by their ({\tt yyyy-mm-dd}) date,
and there are also builtin labels for the \hyperref[ch:calendar]{calendar},
\hyperref[ch:timeline]{timeline}, and \hyperref[ch:issues]{issues} chapters.

%-----------------------------------------------------------------------------------------

\logday[2] % add entry, skipping 2 days between this & last entry
Here is a checklist:
\begin{checklist}
	\item* Checked item 1
	\item* Checked item 2
	\item! Crossed item
	\item Unchecked item
\end{checklist}

%-----------------------------------------------------------------------------------------

\logday[1]
A core feature of this journal is the {\tt tasklist} envionment:
% The tasklist is a fancy 'description' environment. You can use a number of commands
% inside a tasklist, namely \todo, \qstn, \ans, \ex, and \read. Here is an overview of
% the syntax:
%
%	\begin{tasklist}
%		\todo[le18a][\#~2] "Be" attentive -- unresolved todo
%		     ^^^^^^^^^^^^^ ----------------- optional citation & suffix
%                          ^^^^ ------------ issue list entry line content
%		\todo*[le18a][\#~2] "Be"
%		     ^ ----------------------------- 'resolved' (ticked & removed from issue list)
%		\todo?[le18a][\#~2] "Be"
%		     ^ ----------------------------- 'low-priority' (separate issue list)
%
%		\qstn*[ul12b] "How?" --------------- question: same arguments as \todo
%
%		\ex[le18a][1,3--6] "ez." ----------- exercise task: same arguments as \todo
%		\ex?*[le18a][1,3--6] "ez."
%		   ^^ ------------------------------ special combination: see below
%		\read[le18a][1,5--8] ":)" ---------- reading task: same arguments as \ex
%
%		\ans[q:how] No --------------------- answer
%		    ^^^^^^^ ------------------------ optional ref to \label-ed question
%	\end{tasklist}
%
% The issue list entry line (enclosed in ") is optional, defaulting to the empty string,
% but it is nice to see a preview of the issue in its issue list entry. It is
% automatically truncated, so don't worry about its length.
%
% The combination ?* on an \ex or \read task marks it as having been completed on a
% date after it appears. Visually this ticks it in orange instead of blue. The idea is
% that you can use \ex & \read either as a RECORD of daily reading / exericises, or as
% a more specific TO-DO. So the intended semantics are:
%
%	\read*  ------------ done on current day
%	\read   ------------ to do later
%	\read?  ------------ to do later (low-priority)
%	\read?* ------------ to-do completed (on a day after where it appears)
%
\begin{tasklist}
	\todo* "Learn how to use tasklists."
	\todo? "Tick this item off."
	\read "This example journal."
	\read*[baez1994gauge][pp.~10--26] "You can refer to citations."
	\ex[baez1994gauge][\#~1--3] "Exercises work just like reading tasks."

	% Note: the label is best put after the quoted section, or otherwise in the same
	% paragraph as the \qstn command.
	\qstn[baez1994gauge] "Why did I use this book as an example?" \label{qstn:why}
		It just makes no sense.

	% Note: you can click the arrow which is typeset by this answer to jump to the
	% question; the question can have multiple partial answers, so is not marked as
	% 'resolved' until you do so manually.
	\ans[qstn:why] I probably just copied a piece of my actual reference list.
\end{tasklist}

%-----------------------------------------------------------------------------------------

\logday
\begin{theorem}
The following theorem environments are set up: {\tt theorem, definition, lemma,
proposition, corollary}
\end{theorem}
\begin{proof}
Try them!
\end{proof}

%-----------------------------------------------------------------------------------------

\logday[1]
In case you missed it in the source code, there is a small idiosyncrasy with
how we use \verb|\read| and \verb|\ex| both as daily \emph{records} of completed
reading / exercises, and as specialized \emph{to-do}s.

\vspace{1em}
As unresolved to-dos, they can appear in the high- and low-priority forms:
\begin{tasklist}
	\read "High priority reading to-do."
	\read? "Low priority reading to-do."
\end{tasklist}

There are two distinct 'ticked' forms: one denotes reading completed on the same day as it
appears, while the other denotes a reading to-do that was resolved on a later date than
that where it appears:
\begin{tasklist}
	\read* Log of reading done today.
	\read?* Reading to-do that has been completed on a date after today.
\end{tasklist}

{\color{black!20}\hrule}

\vspace{1em}
The next entry will proceed a page break since it is in the new year.

%-----------------------------------------------------------------------------------------

\logday
That's all of the important bits. Enjoy --- I hope you get as much utility from this
template as I did!

\bibliography{references}
\end{document}
