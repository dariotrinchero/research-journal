\documentclass{journal}

%-----------------------------------------------------------------------------------------
% Custom definitions / imports
%-----------------------------------------------------------------------------------------

% TODO add custom definitions & imports here

%-----------------------------------------------------------------------------------------
% Journal setup
%-----------------------------------------------------------------------------------------

\name{Research Journal}
\project{PhD Journalism}
\subject{The use of LaTeX typesetting for journalism} % only in PDF file metadata
\author{Alice Atwood \and Bob Benston}

%-----------------------------------------------------------------------------------------
% Front matter
%-----------------------------------------------------------------------------------------

% The following line mocks current date so that template compiles to a
% consistent output for demonstration purposes.
\SetDate[10/01/2023] % TODO DELETE THIS LINE

\begin{document}

% Title page
\maketitle

% Optional table of contents; two-column layout unless "widetoc" class option given.
\tableofcontents

% Optional "Calendar" section with links to journal entries & current day highlighted.
\calendar[11-01]{2022}{2025} % syntax is \calendar[first-day]{start-year}{end-year}
% TODO add option to change end day

% Optional "Planned Timeline" section with Gantt chart(s) from figures/gantt.tex.
\timelines % \timelines[path] gives custom path to Gantt chart definitions

% Optional "Open Issues" list of unresolved questions & TODOs (see example entries for
% syntax of adding & resolving issues).
\listofissues

%-----------------------------------------------------------------------------------------
% Entries
%-----------------------------------------------------------------------------------------

% Set starting date for journal entries; date here should be day BEFORE first entry.
\logstart{2022-12-02} % use yyyy-mm-dd format

\logday
Add entries with \verb|\logday|. Note that days are automatically
\hyperref[2022-12-03]{labeled} by their date. You can skip days using
\verb|\logday[<days>]|.

% TODO add entries showing off features

\bibliography{references}
\end{document}
